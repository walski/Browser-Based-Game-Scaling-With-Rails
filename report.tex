\documentclass[1p]{scrartcl}

\usepackage[utf8]{inputenc}
\usepackage{graphicx}
\usepackage{url}
\usepackage{ngerman}
\usepackage{times}

\title{Skalierbarkeit von Webanwendungen\\mit Ruby on Rails\\
\small{im Vergleich zu einem bestehenden\\Massively Multiplayer Online-Browserspiel}\\
\small{Bachelor Report - B.Sc. Informatik}\\
}

\author{Universität Bremen / AG Rechnernetze\\Thorben Schröder (walski@tzi.de)\\Betreuer: Carsten Bormann (cabo@tzi.de)}

\date{31. Mai 2009}

\KOMAoptions{DIV=16}

\begin{document}

\maketitle

\tableofcontents
\newpage

  \section{Lastmodell}
  \subsection{Aufbau}
  Aktionen im Spiel können von den Spielern nur durchgeführt werden, wenn sich
  diese vorher am Spiel angemeldet haben. Auch das Lastmodell muss diese
  Anmeldevorgänge abbilden. Wichtig ist dies vor allem für den Beginn einer
  Lastsimulation. Es entspräche nicht dem realen Benutzerverhalten, wenn sich
  schlagartig eine riesige Zahl an Benutzern am Spiel anmeldet. Daher ist das
  Lastmodell in zwei Phasen aufgeteilt:
  \begin{itemize}
    \item Die Aufwärmphase
    \item Die Lastphase
  \end{itemize}
  
  \subsubsection{Aufwärmphase}
  Die Aufwärmphase ist ein Zeitraum zu Beginn eines Lasttests, in dem nach und
  nach Anmeldeanfragen als erste Aktion jeder Folge, an das Spiel gestellt 
  werden und jede dieser angemeldeten Session auch schon die normalen, aus den 
  Markov Ketten generierten, Spielaktionen durchführt. Jede Folge beginnt also 
  mit einem Anmeldevorgang, diese starten aber zeitlich zufällig versetzt.
  
  Wichtig ist, dass während dieser Phase noch keine Lastdaten gesammelt werden,
  da die Last noch stetig zunimmt. Sobald aber ein festgesetzter Prozentsatz der
  simulierten Sessions am System angemeldet ist, markiert dies den Übergang in
  die Lastphase.
  
  \subsubsection{Lastphase}
  Die Lastphase ist die Phase der Lastsimulation, in der die Lastmessung 
  stattfindet. Auch in ihr ist die Zahl der gleichzeitig am Spiel angemeldeten
  Spieler (Concurrent User Sessions - CUS) nicht konstant, ist aber auch nicht
  mehr stetig ansteigend. Dieser erste Anstieg, quasi das Anlaufen des Systems,
  ist in der Aufwärmphase abgeschlossen. Die Lastphase simuliert also einen
  glaubhaften Ausschnitt aus dem Betrieb eines Spiels in der die verschiedenen
  Spieler bereits unterschiedlich lange angemeldet sind und daher an 
  völlig verschiedenartige Aktionen durchführen.
  
  \section{Lastsimulation}
  \subsection{Evaluation bestehender Werkzeuge}
  \subsubsection{Faban}
  \url{http://faban.sunsource.net/}
  Eine Java Benchmark Lösung mit der es, anders als mit einfachen Tools wie
  beispielsweise AB oder httperf, möglich sein soll auch komplexere Benchmarks
  durchzuführen. Die Benchmarks werden dabei nicht konfiguriert sondern können
  in Java programmiert werden. Faban sorgt dann mit einer Infrastruktur aus 
  einem ``Master Server'' und beliebig vielen, auch auf anderen Hosts als der
  Master Server liegenden, Agenten dafür, dass dieser Benchmark aufgeführt und
  die Ergebnisse der einzelnen Agenten wieder zusammengeführt werden.
  
  Die Konfiguration und Benutzung von Faban gestaltet sich schwierig. Nachdem 
  dann auch aus der Dokumentation hervorgeht, dass die Implementierung von
  zu verfolgenden Klickpfaden per Markovketten ein Erweitern des Faban Systems
  unumgänglich machen wird das Simulieren von Lasten mit Faban zunächst zu 
  Gunsten der Evaluierung anderer Systeme aufgegeben.
  
  \subsubsection{Apache Bench - ab}
  \url{http://httpd.apache.org/docs/2.0/programs/ab.html}
  Apache Bench (\textit{ab}) ist ein Kommandozeilen Programm, dass es auf einfache Weise
  erlaubt einen Benchmark auf eine einzelne Ressource per HTTP durchzuführen.
  \textit{ab} misst dabei vor allem die Auslieferungszeit der Ressource und errechnet
  einige Hilfswerte wie zum Beispiel die bedienten Anfragen pro Sekunde.
  Auch wenn \textit{ab} die vorgestellte Aufgabe gut erfüllt eignet es sich nur sehr
  eingeschränkt für das Testen einer Folge von Ressourcen, wie es im Lastmodell
  dieser Arbeit vorgesehen ist. Der einzige Weg mit \textit{ab} solche Folgen 
  abzuarbeiten wäre es für jede Ressource einen neuen \textit{ab} Prozess zu starten, was
  untereinander zu Problemen bei der Synchronisierung führen würde, 
  schwer zu kontrollieren wäre und zudem den Nachteil hätte, dass die Ergebnisse
  schwer zusammenführen sind.
  
  Aus diesen Gründen wird vom Gebrauch von \textit{ab} als Lastsimulator für diese Arbeit
  abgesehen.
  
  \subsubsection{httperf}
  \url{http://www.hpl.hp.com/research/linux/httperf/}
  httperf ist, ähnlich wie \textit{ab}, ein Kommandozweilen Werkzeug zum
  Durchführen von Benchmarks per HTTP. Auch die dabei gemessenen Werte ähneln
  denen von \textit{ab}. Im Gegensatz zu \textit{ab} erlaubt httperf allerdings
  nicht nur das Benchmarken einer einzelnen Ressource, sondern auch das einer
  ganzen Folge von Ressourcen. Die Konfiguration dieser Folgen ist sehr 
  detailliert möglich so sind insbesondere auch Wartezeiten zwischen zwei 
  Aufrufen und das Gleichzeitige Aufrufen mehrerer Ressourcen zu einem Zeitpunkt
  möglich, um so beispielsweise den Aufruf eines HTML Dokuments sowie aller
  darin referenzierten JavaScript-, Bild- und 
  Cascading Style Sheets (CSS)-Dateien zu simulieren.
  
  Mit dieser, in der httperf Dokumentation ``replay'' genannten,
  Möglichkeit Folgen von Ressourcen zu benchmarken wäre es Möglich httperf als
  Lastsimulationswerkzeug im Rahmen dieser Arbeit zu verwenden.
  
    
  \section{Glossar}
  \begin{itemize}
    \item Cascading Style Sheets
    \item HTML
    \item JavaScript
    \item ab
    \item httperf
    \item faban
    \item Markov Kette
    \item Session
    \item Concurrent User Sessions
  \end{itemize}
\end{document}