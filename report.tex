\documentclass[1p]{scrartcl}

\usepackage[utf8]{inputenc}
\usepackage{graphicx}
\usepackage{url}
\usepackage{ngerman}
\usepackage{times}

\title{Skalierbarkeit von Webanwendungen\\mit Ruby on Rails\\
\small{im Vergleich zu einem bestehenden\\Massively Multiplayer Online-Browserspiel}\\
\small{Bachelor Report - B.Sc. Informatik}\\
}

\author{Universität Bremen / AG Rechnernetze\\Thorben Schröder (walski@tzi.de)\\Betreuer: Carsten Bormann (cabo@tzi.de)}

\date{31. Mai 2009}

\KOMAoptions{DIV=16}

\begin{document}

\maketitle

\tableofcontents
\newpage

  \section{Lastmodell}
  \section{Lastsimulation}
  \subsection{Evaluation bestehender Werkzeuge}
  \subsubsection{Faban}
  \url{http://faban.sunsource.net/}
  Eine Java Benchmark Lösung mit der es, anders als mit einfachen Tools wie
  beispielsweise AB oder httperf, möglich sein soll auch komplexere Benchmarks
  durchzuführen. Die Benchmarks werden dabei nicht konfiguriert sondern können
  in Java programmiert werden. Faban sorgt dann mit einer Infrastruktur aus 
  einem ``Master Server'' und beliebig vielen, auch auf anderen Hosts als der
  Master Server liegenden, Agenten dafür, dass dieser Benchmark aufgeführt und
  die Ergebnisse der einzelnen Agenten wieder zusammengeführt werden.
  
  Die Konfiguration und Benutzung von Faban gestaltet sich schwierig. Nachdem 
  dann auch aus der Dokumentation hervorgeht, dass die Implementierung von
  zu verfolgenden Klickpfaden per Markovketten ein Erweitern des Faban Systems
  unumgänglich machen wird das Simulieren von Lasten mit Faban zunächst zu 
  Gunsten der Evaluierung anderer Systeme aufgegeben.
  
    
\end{document}